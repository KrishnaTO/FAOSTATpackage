\documentclass[nojss]{jss}\usepackage{graphicx, color}
%% maxwidth is the original width if it is less than linewidth
%% otherwise use linewidth (to make sure the graphics do not exceed the margin)
\makeatletter
\def\maxwidth{ %
  \ifdim\Gin@nat@width>\linewidth
    \linewidth
  \else
    \Gin@nat@width
  \fi
}
\makeatother

\IfFileExists{upquote.sty}{\usepackage{upquote}}{}
\definecolor{fgcolor}{rgb}{0.2, 0.2, 0.2}
\newcommand{\hlnumber}[1]{\textcolor[rgb]{0,0,0}{#1}}%
\newcommand{\hlfunctioncall}[1]{\textcolor[rgb]{0.501960784313725,0,0.329411764705882}{\textbf{#1}}}%
\newcommand{\hlstring}[1]{\textcolor[rgb]{0.6,0.6,1}{#1}}%
\newcommand{\hlkeyword}[1]{\textcolor[rgb]{0,0,0}{\textbf{#1}}}%
\newcommand{\hlargument}[1]{\textcolor[rgb]{0.690196078431373,0.250980392156863,0.0196078431372549}{#1}}%
\newcommand{\hlcomment}[1]{\textcolor[rgb]{0.180392156862745,0.6,0.341176470588235}{#1}}%
\newcommand{\hlroxygencomment}[1]{\textcolor[rgb]{0.43921568627451,0.47843137254902,0.701960784313725}{#1}}%
\newcommand{\hlformalargs}[1]{\textcolor[rgb]{0.690196078431373,0.250980392156863,0.0196078431372549}{#1}}%
\newcommand{\hleqformalargs}[1]{\textcolor[rgb]{0.690196078431373,0.250980392156863,0.0196078431372549}{#1}}%
\newcommand{\hlassignement}[1]{\textcolor[rgb]{0,0,0}{\textbf{#1}}}%
\newcommand{\hlpackage}[1]{\textcolor[rgb]{0.588235294117647,0.709803921568627,0.145098039215686}{#1}}%
\newcommand{\hlslot}[1]{\textit{#1}}%
\newcommand{\hlsymbol}[1]{\textcolor[rgb]{0,0,0}{#1}}%
\newcommand{\hlprompt}[1]{\textcolor[rgb]{0.2,0.2,0.2}{#1}}%

\usepackage{framed}
\makeatletter
\newenvironment{kframe}{%
 \def\at@end@of@kframe{}%
 \ifinner\ifhmode%
  \def\at@end@of@kframe{\end{minipage}}%
  \begin{minipage}{\columnwidth}%
 \fi\fi%
 \def\FrameCommand##1{\hskip\@totalleftmargin \hskip-\fboxsep
 \colorbox{shadecolor}{##1}\hskip-\fboxsep
     % There is no \\@totalrightmargin, so:
     \hskip-\linewidth \hskip-\@totalleftmargin \hskip\columnwidth}%
 \MakeFramed {\advance\hsize-\width
   \@totalleftmargin\z@ \linewidth\hsize
   \@setminipage}}%
 {\par\unskip\endMakeFramed%
 \at@end@of@kframe}
\makeatother

\definecolor{shadecolor}{rgb}{.97, .97, .97}
\definecolor{messagecolor}{rgb}{0, 0, 0}
\definecolor{warningcolor}{rgb}{1, 0, 1}
\definecolor{errorcolor}{rgb}{1, 0, 0}
\newenvironment{knitrout}{}{} % an empty environment to be redefined in TeX

\usepackage{alltt}
\usepackage{url}
\usepackage[sc]{mathpazo}
\usepackage{geometry}
\geometry{verbose,tmargin=2.5cm,bmargin=2.5cm,lmargin=2.5cm,rmargin=2.5cm}
\setcounter{secnumdepth}{2}
\setcounter{tocdepth}{2}
\usepackage{breakurl}
\usepackage{draftwatermark}


%% almost as usual
\author{Michael. C. J. Kao\\ Food and Agriculture Organization \\of
  the United Nations}
\title{A Package for Transparent and Reproducible\\ Statistics: Package \pkg{FAOSTAT}}



%% for pretty printing and a nice hypersummary also set:
\Plainauthor{Michael. C. J. Kao} %% comma-separated
\Plaintitle{A Transparent and Reproducible Package for Official
  Statistics: Package FAOSTAT} %% without formatting
\Shorttitle{\pkg{FAOSTAT}: A Transparent and Reproducible Package for Official
  Statistics} %% a short title (if necessary)

%% an abstract and keywords
\Abstract{

  The aim of this document is to introduce the FAOSTAT package
  developed by the Food and Agricultural Organization of the United
  Nations which serves as an open gate way to an extensive library of
  agricultural statistics on FAOSTAT and the core back bone for the
  production of the Statistcal Yearbook.

  Dealing with official statistics is a very tedious and complex task
  which is usually overlooked. This paper will address some of the
  common problems we have encountered and at the same time demonstrate
  how some of these obstacles can be alleviated by the package and
  provide a framework for reproducible statistics.

  %% Reproducible research are common in academic.

  The use of open source software \proglang{R} and \LaTeX{} brings
  tremendous amount of benefits, speeding up the production process
  and open up the data and methodology to the general public.

  In this small paper we will illustrate the production process and
  demonstrate how the use of the package can increase transparency and
  sustainability. We will also point out details along the process
  which are typically overseen by analysts and researchers.

}
\Keywords{R, Official Statistics}
\Plainkeywords{R, Official Statistics} %% without formatting
%% at least one keyword must be supplied

%% publication information
%% NOTE: Typically, this can be left commented and will be filled out by the technical editor
%% \Volume{50}
%% \Issue{9}
%% \Month{June}
%% \Year{2012}
%% \Submitdate{2012-06-04}
%% \Acceptdate{2012-06-04}

%% The address of (at least) one author should be given
%% in the following format:
\Address{
  Michael C.J. Kao\\
  Economics and Social Statistics Division\\
  Economic and Social Development Department\\
  United Nations Food and Agriculture Organization\\
  Viale delle Terme di Caracalla 00153 Rome, Italy\\
  E-mail: \email{michael.kao@fao.org}
  % URL: \url{http://eeecon.uibk.ac.at/~zeileis/}
}
%% It is also possible to add a telephone and fax number
%% before the e-mail in the following format:
%% Telephone: +43/512/507-7103
%% Fax: +43/512/507-2851



%% end of declarations %%%%%%%%%%%%%%%%%%%%%%%%%%%%%%%%%%%%%%%%%%%%%%%


\begin{document}

%% include your article here, just as usual
%% Note that you should use the \pkg{}, \proglang{} and \code{} commands.



\section{Introduction}
The idea of utilizing \proglang{R} and \LaTeX{} for the production of
the FAO statistical yearbook was initiated by Adam Prakash and Mattieu
Stigler in the ESS division of the Food and Agricultural Organization
of the United Nations. The initiative was taken in order to replace
the labour intensive work with a streamline system which integrates
data extraction, manipulation, statistical graphics and tables into
one single comprehensive system.

This paper will demonstrate how the FAOSTAT package is used to
download, and process data under the framework of the FAO Statistical
Yearbook. The goal is to provide a framework for users to access, and
reproduce the statistics released in the publication and also easy the
pain for dealing with official statistics from various sources.  By
providing the data and methodology to the public, we hope to receive
feedbacks from domain experts and professionals in order to improve
the system and at the same time to promote research and analysis in
the field which will support for better policies and decision.


First, we will demonstration the usage of the \code{getFAOtoSYB} and
\code{getWDItoSYB} functions to download data from the FAO FAOSTAT and
the World Bank WDI API. This is then followed by the demonstration of
the \code{translateCountryCode} and \code{mergeSYB} function to merge
data from various sources and address some of the common
complications. Finally we provide examples of how the aggregates can
be computed using the function\code{aggRegion} for different
composition.

\section{Motivation}
Compiling hundreds of statistics from different sources under
traditional approach such as Excel can be very labour intensive and
error proned. Furthermore, the knowledge and the experience is almost
impossible to sustain in the long run resulting in inconsistent
results and treatments over time. As a result, the ESS took the
initiative to use \proglang{R} and \proglang{\LaTeX} as the new
architecture for a sustainable and cost-effective way to produce the
statistical yearbook. This approach increases the sustainability and
coherence of the publication as all the data manipulation, and
exceptions are recorded in the source code.

In addition to these working motives, the use of R enables the data
generated by the publication to be reproducible and readily accessible
to researchers and analysts around the world. This open-data philosophy
has proven to create tremendous amount of benefits for both the user
and the data provider. We hope that this initiative will increase the
visibility of agricultural related statistics and spark more research
and analysis which the organization and its beneficiaries will gain.

%% After a little more than a year, the architecture has not only become
%% the standard for producing the global statistical yearbook and the
%% pocket book; but the same code is used as template to generate five
%% other regional books. The gain in efficiency was unimaginable with
%% three individual is sufficient to generate four publication including
%% the text within a year and potentially raised to seven this year.

Reproducibility is the norm in academics, this property allows one to
verify, improve and reproduce the research for future use. We believe
that a publication such as the statistical yearbook which publishes
statistics and aggregates should be examined under the same
transparency standard. The publication of the methodology is equally
important as the statistics itself.


%% Talk about the similarity about UNDP and ISO3

%% UNESCO - country name
%% UNICEF - ISO3/UNDP?
%% ILO - ISO3
%% WHO - country name/ISO3

The package can be installed from the CRAN repository just like all
other R packages, and this documentation is also the vignette of the
package.

\begin{knitrout}
\definecolor{shadecolor}{rgb}{0.969, 0.969, 0.969}\color{fgcolor}\begin{kframe}
\begin{alltt}
\hlfunctioncall{if}(!\hlfunctioncall{is.element}(\hlstring{"FAOSTAT"}, \hlfunctioncall{.packages}(all.available = TRUE)))
   \hlfunctioncall{install.packages}(\hlstring{"FAOSTAT"})
\hlfunctioncall{library}(FAOSTAT)
\hlcomment{## vignette("FAOSTAT", package = "FAOSTAT")}
\end{alltt}
\end{kframe}
\end{knitrout}



\section{Download data from FAOSTAT}

FAOSTAT is the largest agricultural database in the world, it contains
data from land productivity to agricultural production and trade
dating back from 1960 to the most recent available data. Detailed
information on meta data, methods and standards can be found on the
official website of FAOSTAT
\url{http://faostat3.fao.org/home/index.html} and the Statistics
Division (ESS) \url{http://www.fao.org/economic/ess/en/}.

In order to access the data using the FAOSTAT API, the domain, item
and element code of the indicator of interest is required. They are
defined as:

\begin{itemize}
  {\setlength\itemindent{48pt}\item[Domain Code]: \hfill\\ The domain
    associated with the data. For example, production, trade, prices
    etc.}  {\setlength\itemindent{30pt}\item[Item Code]: \hfill\\
    These are the codes relating to the commodity or product group
    such as wheat, almonds, and aggregated item like "total cereals".}
  {\setlength\itemindent{49pt}\item[Element Code]: \hfill\\Lastly,
    this is the quantity/unit or data collection type associated with
    the commodity. Typical elements are quantity, value or
    production/extraction rate.}
\end{itemize}

An interactive function \code{FAOsearch} has been provided for the
user to identify the respective codes. An object \code{.LastSearch}
will be assigned after the search and can be used by both the
\code{getFAO} and \code{getFAOtoSYB} functions as the sole argument to
download the data.

\begin{knitrout}
\definecolor{shadecolor}{rgb}{0.969, 0.969, 0.969}\color{fgcolor}\begin{kframe}
\begin{alltt}
\hlcomment{## Use the interective function to search the code.}
\hlfunctioncall{FAOsearch}()

\hlcomment{## Use the result of the search to download the data.}
test = \hlfunctioncall{getFAO}(query = .LastSearch)
\end{alltt}
\end{kframe}
\end{knitrout}



The \code{getFAOtoSYB} is a wrapper for the \code{getFAO} to batch
download data, it supports error recovery and stores the status of the
download. The function also splits the data downloaded into entity
level and regional aggregates, saving time for the user. Query results
from \code{FAOsearch} can also be used.

\begin{knitrout}
\definecolor{shadecolor}{rgb}{0.969, 0.969, 0.969}\color{fgcolor}\begin{kframe}
\begin{alltt}
\hlcomment{## A demonstration query}
FAOquery.df = \hlfunctioncall{data.frame}(varName = \hlfunctioncall{c}(\hlstring{"arableLand"}, \hlstring{"cerealExp"}, \hlstring{"cerealProd"}),
                         domainCode = \hlfunctioncall{c}(\hlstring{"RL"}, \hlstring{"TP"}, \hlstring{"QC"}),
                         itemCode = \hlfunctioncall{c}(6621, 1944, 1717),
                         elementCode = \hlfunctioncall{c}(5110, 5922, 5510),
                         stringsAsFactors = FALSE)

\hlcomment{## Download the data from FAOSTAT}
FAO.lst = \hlfunctioncall{with}(FAOquery.df,
    \hlfunctioncall{getFAOtoSYB}(name = varName, domainCode = domainCode,
                itemCode = itemCode, elementCode = elementCode,
                useCHMT = TRUE, outputFormat = \hlstring{"wide"}))
\end{alltt}
\end{kframe}
\end{knitrout}


The object returned is a list of length three, these are entity level
data, aggregates and the download status. The function supports both
long and wide format.

In some cases multiple China are provided, for example the trade
domain provides data on China (41), Taiwan (214) and China plus Taiwan
(357). The \code{CHMT} function avoids double counting if multiple
China are detected by removing the aggregated if detected. The default
in \code{getFAOtoSYB} is to use \code{CHMT} when possible. Otherwise,
the \code{FAOcheck} function can be used to sanitize the data.

\begin{knitrout}
\definecolor{shadecolor}{rgb}{0.969, 0.969, 0.969}\color{fgcolor}\begin{kframe}
\begin{alltt}
FAOchecked.df = \hlfunctioncall{FAOcheck}(var = FAOquery.df$varName, year = \hlstring{"Year"},
    data = FAO.lst$entity)
\end{alltt}
\end{kframe}
\end{knitrout}



%% Need the name of the aggregates.

\section{Download data from World Bank}
The World Bank also provides an API where data from the World Bank and
various international organization are made public. More information
about the data and the API can be found at
\url{http://data.worldbank.org/}

The author is aware of the \pkg{WDI} package, but we have wrote this
function before the recent update of the package with additional
functionalities. We have plans to integrate with the \pkg{WDI} package
to avoid confusion for the users.


\begin{knitrout}
\definecolor{shadecolor}{rgb}{0.969, 0.969, 0.969}\color{fgcolor}\begin{kframe}
\begin{alltt}
\hlcomment{## Download World Bank data and meta-data}
WB.lst = \hlfunctioncall{getWDItoSYB}(indicator = \hlfunctioncall{c}(\hlstring{"SP.POP.TOTL"}, \hlstring{"NY.GDP.MKTP.CD"}),
                     name = \hlfunctioncall{c}(\hlstring{"totalPopulation"}, \hlstring{"GDPUSD"}),
                     getMetaData = TRUE, printMetaData = TRUE)
\end{alltt}
\end{kframe}
\end{knitrout}


The output is similar to the object generated by \code{getFAOtoSYB}
except that if the arguement \code{getMetaData} is specified as TRUE
then the meta data is also downloaded and saved.

One point to note here, it is usually unlikely to reconstruct the
world aggregates provided by the World Bank based on the data
provided. The reason is that the aggregate contains Taiwan when
available, yet the statistics for Taiwan are not published.



\section{Merge data from different sources}
Merge is a typical data manipulation step in daily work yet a
non-trivial exercise especially when working with different data
sources. The built in \textit{mergeSYB} function enables one to merge
data from different sources as long as the country coding system is
identified. Currently the following country coding translation are
supported and included in the internal data set FAOcountryProfile of
the package:


\begin{itemize}
  \item United Nations M49 country standard [UN\_CODE]\\
    \url{http://unstats.un.org/unsd/methods/m49/m49.htm}
  \item FAO country code scheme [FAOST\_CODE]\\
    \url{http://termportal.fao.org/faonocs/appl/}
  \item FAO Global Administrative Unit Layers (GAUL).[ADM0\_CODE]
  \item ISO 3166-1 alpha-2 [ISO2\_CODE]\\
    \url{http://en.wikipedia.org/wiki/ISO\_3166-1\_alpha-2}
  \item ISO 3166-1 alpha-2 (World Bank) [ISO2\_WB\_CODE]\\
    \url{http://data.worldbank.org/node/18}
  \item ISO 3166-1 alpha-3 [ISO3\_CODE]\\
    \url{http://en.wikipedia.org/wiki/ISO\_3166-1\_alpha-3}
  \item ISO 3166-1 alpha-3 (World Bank) [ISO3\_WB\_CODE]\\
    \url{http://data.worldbank.org/node/18}
\end{itemize}

Data from any sources employ country classification listed above can
be supplied to \code{mergeSYB} in order to obtain a single merged
data. However, the column name of the country coding scheme is
required to be the same as the name in square bracket, the
responsibility of identifying the coding system lies with the user.

Nevertheless, often only the name of the country is provided and thus
merge is not possible or inaccurate based on names. We have provided a
function to obtain country codes based on the names matched. In order
to avoid matching with the wrong code, the function only attempts to
fill in countries which have exact match.

\begin{knitrout}
\definecolor{shadecolor}{rgb}{0.969, 0.969, 0.969}\color{fgcolor}\begin{kframe}
\begin{alltt}
\hlcomment{## Just a demonstration}
Demo = WB.lst$entity[, \hlfunctioncall{c}(\hlstring{"Country"}, \hlstring{"Year"}, \hlstring{"totalPopulation"})]
demoResult = \hlfunctioncall{fillCountryCode}(country = \hlstring{"Country"}, data = Demo,
    outCode = \hlstring{"ISO2_WB_CODE"})

\hlcomment{## Countries have not been filled in.}
\hlfunctioncall{unique}(demoResult[\hlfunctioncall{is.na}(demoResult$ISO2_WB_CODE), \hlstring{"Country"}])
\end{alltt}
\begin{verbatim}
## [1] "China" "Sudan"
\end{verbatim}
\end{kframe}
\end{knitrout}


We have not implemented a regular expression match for the
identification reason listed below. From the above example we can see
that both China and Sudan are not filled in, the identification of
Sudan prior to 2011 and China should be carefully examined.


Below we list some commonly observed problem when merging data from
different sources.

\subsection{Identification problem}
Due to the fact that different organization are bounded by different
political agenda the users need to be aware of the precise definition
and legal recognition of countries.

Take example, the China provided by the World Bank does not include
Taiwan, Hong Kong and Macao. On the other hand, FAO provides not only
a single China (FAO = 41), but also China plus Taiwan (FAO = 357)
depending on the context. In addition, it is common to observed
statistics for China (ISO2 = CN or ISO3 = CHN) which includes Taiwan,
Hong Kong and Macao. The default trnslates China from other country
coding scheme to Mainland China (FAO = 41) and is not matched in
\code{fillCountryCode}.

\subsection{Representation problem}
Moreover, the situation is further complicated by disputed territories
or economic union such as Kosovo and Belgium-Luxembourg which does not
have representation under particular country coding system.

\begin{knitrout}
\definecolor{shadecolor}{rgb}{0.969, 0.969, 0.969}\color{fgcolor}\begin{kframe}
\begin{alltt}
\hlcomment{## Countries which are not listed under the ISO2 international standard.}
FAO.df = \hlfunctioncall{translateCountryCode}(data = FAOchecked.df, from = \hlstring{"FAOST_CODE"},
    to = \hlstring{"ISO2_CODE"})
\end{alltt}
\begin{verbatim}
## 
## NOTE: Please make sure that the country are matched according to their definition
\end{verbatim}


{\ttfamily\noindent\color{warningcolor}{\#\# Warning: The following entries does not have 'ISO2\_CODE' available}}\begin{verbatim}
##     FAOST_CODE ISO2_CODE
## 29          15      <NA>
## 56         259      <NA>
## 58         351      <NA>
## 62         357      <NA>
## 95          62      <NA>
## 193        147      <NA>
##                                            OFFICIAL_FAO_NAME
## 29                                        Belgium-Luxembourg
## 56                                           Channel Islands
## 58  China (China mainland, Hong Kong SAR, Macao SAR, Taiwan)
## 62                            China (China mainland, Taiwan)
## 95                                              Ethiopia PDR
## 193                                  the Republic of Namibia
\end{verbatim}
\begin{alltt}

\hlcomment{## Countries which are not listed under the UN M49 system.}
WB.df = \hlfunctioncall{translateCountryCode}(data = WB.lst$entity, from = \hlstring{"ISO2_WB_CODE"},
    to = \hlstring{"UN_CODE"})
\end{alltt}
\begin{verbatim}
## 
## NOTE: Please make sure that the country are matched according to their definition
\end{verbatim}


{\ttfamily\noindent\color{warningcolor}{\#\# Warning: The following entries does not have 'UN\_CODE' available}}\begin{verbatim}
##     ISO2_WB_CODE UN_CODE OFFICIAL_FAO_NAME
## 155           KV      NA            Kosovo
\end{verbatim}
\end{kframe}
\end{knitrout}



\subsection{Transition problem}
Finally, the discontinuity and transition of countries further
increases the complexity of the data. The South Sudan was recognized
by the United Nations on the 9th of July 2011, however, the statistics
reported by The republic of the Sudan in the same year can also
includes data for South Sudan thus failing the mutually exclusive
test. Moreover, sources which uses ISO standard country code have no
way to distinguish between the new and old Sudan (SD and SDN are used
for both entity) which causes problem in merge with country system
that distingiushes the entity.

Finally, if historical aggregates are computed then a region
composition which does not backtrack in time will result in an
aggregate which is incorrect. For more details about historical and
transitional countries please refer to
\url{http://unstats.un.org/unsd/methods/m49/m49chang.htm}

Given the lack of an internationally recognized standard which
incorporates all these properties, we suggests the use of the FAO
country standard and region profile shipped with the package which
addresses most of these problems.

\begin{knitrout}
\definecolor{shadecolor}{rgb}{0.969, 0.969, 0.969}\color{fgcolor}\begin{kframe}
\begin{alltt}
merged.df = \hlfunctioncall{mergeSYB}(FAOchecked.df, WB.lst$entity, outCode = \hlstring{"FAOST_CODE"})
\end{alltt}
\begin{verbatim}
## 
## NOTE: Please make sure that the country are matched according to their definition
## 
## 
## NOTE: Please make sure that the country are matched according to their definition
\end{verbatim}
\end{kframe}
\end{knitrout}



%% \section{Computing growth, and other derivatives}
%% There are two types of growth rate shipped with the package, the least
%% squares growth rate and the geometric growth rate. The least squares
%% growth rate is used when the time series is of sufficient length. The
%% default is at least 5 usable observations, however if the time series
%% is sparse and more than 50\% of the data are missing than the robust
%% regression is used.


%% Check how to fix the problem with zero when computing least squares
%% growh rate

%% Catch the error and examine the problem

%% Rewrite the lsgr, since if there are gaps in the data then the all
%% NA statement is not covered. The lsgr should not be computed by
%% rolling.

%% The loop needs to be taken outside.


\section{Computing regional or economical aggregates}
Aggregation is another data manipulation step that is commonly over
seen. The result can vary due to the differences between the regional
composition and the set of countries used. Furthermore, it is
complicated by the amount of missing values which can render the
aggregates incomparable. Given the missing values and diverging
country sets, aggregation can only serve as approximates in order to
inform the general situation of the region. The following rules are
implemented to ensure the aggregates computed are meaningful and
comparable.


\begin{itemize}
  \item A minimum threshold in which the data must be present, the
    default is 65\% for every individual year.
  \item The number of reporting entities must be similar over the
    years. It does not make sense to compare aggregates of 1995 and
    2000 if the number of reporting countries differ vastly, the
    default tolerance is 15.
\end{itemize}

In addition, historical countries are aggregated to ensure
comparability over time. For example, The Former Soviet Union is not
part of the current definition of the M49 standards, nevertheless,
it would be ignorant to omit it from the aggregation.

\begin{knitrout}
\definecolor{shadecolor}{rgb}{0.969, 0.969, 0.969}\color{fgcolor}\begin{kframe}
\begin{alltt}
\hlcomment{## Compute aggregates under the FAO continental region.}
relation.df = FAOregionProfile[, \hlfunctioncall{c}(\hlstring{"FAOST_CODE"}, \hlstring{"FAO_MACRO_REG"})]

FAOregion.df = \hlfunctioncall{aggRegion}(data = merged.df, relationDF = relation.df,
                         aggVar = \hlfunctioncall{c}(\hlstring{"arableLand"}, \hlstring{"cerealExp"}, \hlstring{"cerealProd"},
                                    \hlstring{"totalPopulation"}, \hlstring{"GDPUSD"}),
                         aggMethod = \hlfunctioncall{rep}(\hlstring{"sum"}, 5),
                         unspecifiedCode = \hlstring{"Unspecified"})

\hlcomment{## Compute aggregates under the UNSD M49 continental region.}
relation.df = FAOregionProfile[, \hlfunctioncall{c}(\hlstring{"FAOST_CODE"}, \hlstring{"UNSD_MACRO_REG"})]

UNregion.df = \hlfunctioncall{aggRegion}(data = merged.df, relationDF = relation.df,
                        aggVar = \hlfunctioncall{c}(\hlstring{"arableLand"}, \hlstring{"cerealExp"}, \hlstring{"cerealProd"},
                                   \hlstring{"totalPopulation"}, \hlstring{"GDPUSD"}),
                        aggMethod = \hlfunctioncall{rep}(\hlstring{"sum"}, 5),
                        unspecifiedCode = \hlstring{"Unspecified"})

\end{alltt}
\end{kframe}
\end{knitrout}


\section{Conclusion}




\section*{Acknowledgement}
The author owes a great debt to Fillipo Gheri, Adam Prakash, Guido
Barbaglia, Amy Heyman, Amanda Gordon, Jacques Joyeux, and Markus
Gesmann for their contribution to the pacakge, whom the package would
not exist without their expertise.

The author would also like to express his profound gratitude to the
directors Pietro Gennari and Josef Schimidhuber and the entire ESS
division for their support, experience and every little thing they
have done was greatly appreciated.


%% The author understand that a standardised framework such as SDMX is
%% under way, yet this serves as current solution.

%% These are internal knowledges we would like to share, along with
%% the data we are opening.

%% Develop warning messages for all the disputed territories.

%% We understand that it may be preferable to use the S4 or the R5
%% class to rigorously define the elements.

\end{document}


